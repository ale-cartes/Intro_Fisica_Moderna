\documentclass[letterpaper,11pt]{article}
\oddsidemargin -1.0cm \textwidth 17.5cm

\usepackage[utf8]{inputenc}
\usepackage[activeacute,spanish, es-lcroman]{babel}
\decimalpoint
\usepackage{amsfonts,setspace}
\usepackage{amsmath}
\usepackage{amssymb, amsmath, amsthm}
\usepackage{comment}
\usepackage{float}
\usepackage{amssymb}
\usepackage{dsfont}
\usepackage{anysize}
\usepackage{multicol}
\usepackage{enumerate}
\usepackage{graphicx}
\usepackage[left=1.5cm,top=2cm,right=1.5cm, bottom=1.7cm]{geometry}
\setlength\headheight{1.5em} 
\usepackage{fancyhdr}
\usepackage{multicol}
\usepackage{hyperref}
\usepackage{wrapfig}
\usepackage{subcaption}
\usepackage{siunitx}
\usepackage{cancel}
\usepackage{mdwlist}
\usepackage{svg}
\pagestyle{fancy}
\fancyhf{}
\renewcommand{\labelenumi}{\normalsize\bfseries P\arabic{enumi}.}
\renewcommand{\labelenumii}{\normalsize\bfseries (\alph{enumii})}
\renewcommand{\labelenumiii}{\normalsize\bfseries \roman{enumiii})}


\begin{document}

\fancyhead[L]{\itshape{Facultad de Ciencias F\'isicas y Matem\'aticas}}
\fancyhead[R]{\itshape{Universidad de Chile}}
\rfoot[]{pág. \thepage}

\begin{minipage}{11.5cm}
    \begin{flushleft}
        \hspace*{-0.6cm}\textbf{FI1100 Introducción a la Física Moderna}\\
        \hspace*{-0.6cm}\textbf{Tutor:} Alejandro Cartes
    \end{flushleft}
\end{minipage}

\begin{picture}(2,3)
    \put(366, -10){\includegraphics[scale=0.9]{Imágenes/logo/dfi-fcfm.pdf}}
\end{picture}

\begin{center}
    \LARGE\textbf{Tutoría Examen}\\
    \large{Ec. de Schrödinger y Relatividad Especial}
\end{center}

\vspace{-1cm}
\begin{enumerate}\setlength{\itemsep}{0.4cm}

\item[]

\item Una partícula en movimiento en una dimensión (el eje $x$) se describe por la función de onda:
\[\psi(x) = A \mathop{e^{-b|x|}} = 
\begin{cases}
    A\mathop{e^{-bx}}, & \text{si } x \geq 0\\
    A\mathop{e^{bx}}, & \text{si } x<0
\end{cases}\] 

donde $b = \SI{2.00}{\m^{-1}}$, $A>0$ y el eje x apunta hacia la derecha.
    \begin{enumerate}
        \item Determine $A$ de manera que se normalice la función de onda.
        
        \item Grafique la función de onda.
        
        \item Calcule la probabilidad de encontrar esta partícula en cada una de las siguientes regiones:
            \begin{enumerate}
                \item Dentro de los \SI{50.0}{\cm} del origen
                
                \item Del lado izquierdo del origen
                
                \item Entre $x = \SI{0.500}{\m}$ y $x = \SI{1.00}{\m}$
            \end{enumerate}
    \end{enumerate}

\item Dos naves espaciales, cada una de \SI{100}{\m} de longitud (cuando se miden en reposo) viajan una hacia la otra con velocidades de $0.85c$ relativas a la Tierra.

    \begin{enumerate}
        \item ¿Qué longitud tiene cada nave medida por un observador desde la Tierra?

        \item ¿Qué velocidad tiene cada nave medida por un observador de la otra nave?

        \item ¿Qué longitud tiene cada nave medida por un observador de la otra nave?

        \item En el tiempo $t=0$ se ve desde la Tierra que las dos naves tienen sus extremos frontales en contacto, es decir, comienzan a cruzarse. ¿En qué momento se verán juntos desde la Tierra los extremos posteriores?
    \end{enumerate}

% "C:\Users\aleja\fcfm\Auxiliares y Ayudantías\Intro a la Física Moderna\semestres anteriores\Material Intro Física Moderna Mardones 2020-1\Aux 10 Relatividad Especial\Aux 10 Relatividad Especial.pdf"
\item Al mediodía, un cohete pasa por la Tierra a una velocidad $v=0.8 c$. Tanto los observadores en la nave como los observadores en la Tierra están de acuerdo en que son las 12:00 en punto. 

    \begin{enumerate}
        \item A las 12:30 pm del reloj del cohete, la nave pasa por una estación espacial interplanetaria a una distancia $d$ fija de la Tierra y cuyo reloj está sincronizado con el de la Tierra. ¿Qué hora es en dicha estación?

        \item ¿Cuán lejos de la Tierra está la estación?

        \item Precisamente a las 12:30 pm, según el reloj del cohete, el capitán de la nave manda un reporte por radio de vuelta hacia la Tierra. ¿A qué hora terrestre llega la señal?

        \item La estación en la Tierra responde en forma inmediata. ¿Qué hora marca el reloj de la nave cuando la respuesta llega de vuelta?
    \end{enumerate}

\end{enumerate}
\end{document}
