\documentclass[letterpaper,11pt]{article}
\oddsidemargin -1.0cm \textwidth 17.5cm

\usepackage[utf8]{inputenc}
\usepackage[activeacute,spanish, es-lcroman]{babel}
\decimalpoint
\usepackage{amsfonts,setspace}
\usepackage{amsmath}
\usepackage{amssymb, amsmath, amsthm}
\usepackage{comment}
\usepackage{float}
\usepackage{amssymb}
\usepackage{dsfont}
\usepackage{anysize}
\usepackage{multicol}
\usepackage{enumerate}
\usepackage{graphicx}
\usepackage[left=1.5cm,top=2cm,right=1.5cm, bottom=1.7cm]{geometry}
\setlength\headheight{1.5em} 
\usepackage{fancyhdr}
\usepackage{multicol}
\usepackage{hyperref}
\usepackage{wrapfig}
\usepackage{subcaption}
\usepackage{siunitx}
\usepackage{cancel}
\usepackage{mdwlist}
\usepackage{svg}
\pagestyle{fancy}
\fancyhf{}
\renewcommand{\labelenumi}{\normalsize\bfseries P\arabic{enumi}.}
\renewcommand{\labelenumii}{\normalsize\bfseries (\alph{enumii})}
\renewcommand{\labelenumiii}{\normalsize\bfseries \roman{enumiii})}


\begin{document}

\fancyhead[L]{\itshape{Facultad de Ciencias F\'isicas y Matem\'aticas}}
\fancyhead[R]{\itshape{Universidad de Chile}}
\rfoot[]{pág. \thepage}

\begin{minipage}{11.5cm}
    \begin{flushleft}
        \hspace*{-0.6cm}\textbf{FI1100 Introducción a la Física Moderna}\\
        \hspace*{-0.6cm}\textbf{Tutor:} Alejandro Cartes
    \end{flushleft}
\end{minipage}

\begin{picture}(2,3)
    \put(366, -10){\includegraphics[scale=0.9]{Imágenes/logo/dfi-fcfm.pdf}}
\end{picture}

\begin{center}
    \LARGE\textbf{Tutoría Examen}\\
    \large{Ppio. de Incertidumbre, Ec. de Schrödinger y Relatividad Especial}
\end{center}

\vspace{-1cm}
\begin{enumerate}\setlength{\itemsep}{0.4cm}

\item[]

% http://info.phys.unm.edu/~alandahl/phys262f06/exam3sols.pdf
\item Una partícula de masa $m$ cumple que la incerteza asociada a su posición es igual a su longitud de onda de De Broglie. ¿Cuál es la incerteza fraccional mínima de su rapidez, $\Delta v/v$?

% https://www.lehman.edu/faculty/anchordoqui/gp28solns.pdf
\item Considere el estado basal del oscilador armónico, cuyo potencial está dado por $V(x) = 1/2 m\omega_0^2 x^2$, con $\psi_0(x) = A e^{-ax^2}$ donde $A = (m\omega_0/\pi\hbar)^{1/4}$ y $a=m\omega_0/2\hbar$

Determine la energía del estado basal, $E_0$ utilizando solo el principio de incertidumbre y la expresión general para las incertezas $(\Delta x)^2 = \left<x^2\right> - \left<x\right>^2$

Para ello:
\begin{enumerate}
    \item Muestre que los valores de expectación para la posición y el momentum del estado basal son nulos, $\left<x\right> = \left<p\right> = 0$

    \item Encuentre una expresión para determinar el valor de expectación de la energía, $\left<E\right>$, en términos de $\Delta x$ y $\Delta p$

    \item Usando el resultado anterior, muestre que:
    $$E_0 \leq \frac{\hbar^2}{8m(\Delta x)^2} + \frac{1}{2}m\omega_0^2(\Delta x)^2$$

    \item Determine el valor de $(\Delta x)^2$ correspondiente al mínimo valor de la expresión anterior. Este término permite determinar el menor valor de $\left<E\right>$ consistente con el principio de incertidumbre, respondiendo así a lo pedido.
\end{enumerate}


\item Considere una partícula de masa $m$ en una caja de tamaño $L$ cuyo potencial está descrito por:
\[V(x) = 
    \begin{cases}
        0 & -L/2\leq x\leq L/2\\
        +\infty & \quad \sim
    \end{cases}
\]

    \begin{enumerate}

        \item Determine los niveles de energía que tendrá la partícula
        
        \item Si un átomo de hidrógeno se modela como una caja unidimensional de longitud igual al radio de Bohr, ¿cuál es energía basal del electrón?
    \end{enumerate}

\item Un insecto idealizado, el cual lo podemos aproximar como un punto, vive en el fondo de un agujero de profundidad $L$, por otra parte tenemos un remache que se puede introducir solo a una distancia $a<L$ dentro del agujero. Normalmente uno aseguraría que nuestro insecto está a salvo dentro del agujero, pero estudiemos este problema de manera relativista. Imaginemos que el remache se mueve hacia el agujero con una velocidad de $u=0.9c$

\begin{multicols}{2}
    \begin{enumerate}
        \item ¿Qué ocurre en el sistema de referencia del remache con el insecto?
        \item ¿Qué observa el insecto?
        \item ¿Qué ocurre realmente en el problema?
    \end{enumerate}
    
    \columnbreak
    \begin{figure}[H]
        \centering
        \includegraphics[width=0.5\linewidth]{Imágenes/bug-rivet-paradox_01_r2.png}
    \end{figure}    
\end{multicols}


\item Unos ladrones están escapando de la policía en un auto que se puede mover a una velocidad de $3/4 c$, la policía los persigue en un auto que solo se puede mover a una velocidad de $1/2 c$. El oficial de policía quiere detener a los ladrones disparándole una bala a los neumáticos, la velocidad de la bala (relativa al arma) es de $1/3 c$. ¿Llega la bala a su objetivo de acuerdo a Galileo?¿De acuerdo a Einstein?
\end{enumerate}
\end{document}
