\documentclass[letterpaper,11pt]{article}
\oddsidemargin -1.0cm \textwidth 17.5cm

\usepackage[utf8]{inputenc}
\usepackage[activeacute,spanish, es-lcroman]{babel}
\decimalpoint
\usepackage{amsfonts,setspace}
\usepackage{amsmath}
\usepackage{amssymb, amsmath, amsthm}
\usepackage{comment}
\usepackage{float}
\usepackage{amssymb}
\usepackage{dsfont}
\usepackage{anysize}
\usepackage{multicol}
\usepackage{enumerate}
\usepackage{graphicx}
\usepackage[left=1.5cm,top=1.5cm,right=1.5cm, bottom=1.7cm]{geometry}
\setlength\headheight{1.5em} 
\usepackage{fancyhdr}
\usepackage{multicol}
\usepackage{hyperref}
\usepackage{wrapfig}
\usepackage{subcaption}
\usepackage{siunitx}
\usepackage{cancel}
\pagestyle{fancy}
\fancyhf{}
\renewcommand{\labelenumi}{\normalsize\bfseries P\arabic{enumi}.}
\renewcommand{\labelenumii}{\normalsize\bfseries (\alph{enumii})}
\renewcommand{\labelenumiii}{\normalsize\bfseries \roman{enumiii})}

\begin{document}

\fancyhead[L]{\itshape{Facultad de Ciencias F\'isicas y Matem\'aticas}}
\fancyhead[R]{\itshape{Universidad de Chile}}

\begin{minipage}{11.5cm}
    \begin{flushleft}
        \hspace*{-0.6cm}\textbf{FI1100 Introducción a la Física Moderna}
    \end{flushleft}
\end{minipage}

\begin{picture}(2,3)
    \put(366, -10){\includegraphics[scale=0.9]{Imágenes/logo/dfi-fcfm.pdf}}
\end{picture}

\begin{center}
	\LARGE\textbf{Problemitas de Óptica}
\end{center}

\vspace{-1cm}

\section*{\underline{Reflexión y Refracción}}
\begin{enumerate}\setlength{\itemsep}{0.4cm}

\rfoot[]{pág. \thepage}

\begin{multicols}{2}
\item El \textit{Principio de Fermat} establece que siempre que la luz viaja desde un punto a otro su trayectoria real es la que requiere el intervalo de tiempo más pequeño. Deduzca la Ley de Snell considerando la situación de la figura, para ello
    \begin{enumerate}
        \item Demuestre que el tiempo cuando la luz llega a $Q$ es

        $$t = \frac{n_1\sqrt{a^2+x^2}}{c} + \frac{n_2\sqrt{b^2+(d-x)^2}}{c}$$

        \item Para obtener el valor de $x$ para que $t$ alcance su valor mínimo, derive $t$ con respecto a $x$ e iguale la derivada a cero. Demuestre que se tiene la siguiente relación

        \[\frac{n_1x}{\sqrt{a^2+x^2}} = \frac{n_2(d-x}{\sqrt{b^2+(d-x)^2}}\]

        \item Demuestre que esta expresión da la ley de Snell
    \end{enumerate}

    \columnbreak

    \begin{figure}[H]
        \centering
        \includegraphics[width=0.8\linewidth]{Imágenes/clases/interfaz.png}
    \end{figure}
\end{multicols}

\begin{multicols}{2}
    \item Un material que tiene un índice de refracción $n$ está rodeado por vacío y tiene la forma de un cuarto de círculo de radio $R$. Un rato de luz paralelo a la base del material incide desde la izquierda a una distancia $L$ por encima de la base y emerge desde el material a un ángulo $\theta$. Determine una expresión para $\theta$
    
    \columnbreak
    
    \begin{figure}[H]
        \centering
        \includegraphics[width=0.75\linewidth]{Imágenes/clases/cuarto-circ.png}
    \end{figure}
\end{multicols}

\begin{multicols}{2}
    \item Una fibra óptica tiene un índice de refracción $n$ y un diámetro $d$. Se envía un haz de luz por la fibra a lo largo de su eje, como se muestra en la figura. Si la fibra se encuentra rodeada de aire, determine el mínimo radio exterior $R$ permitido para una curva en la fibra si no ha de escapar luz.

    \columnbreak
    
    \begin{figure}[H]
        \centering
        \includegraphics[width=0.3\linewidth]{Imágenes/clases/fibra.png}
    \end{figure}
\end{multicols}

\begin{multicols}{2}
    \item Un rayo de luz incide sobre la superficie 2 en el ángulo crítico. Determine el ángulo de incidencia $\theta_1$

    \columnbreak
    
    \begin{figure}[H]
        \centering
        \includegraphics[width=0.5\linewidth]{Imágenes/clases/triangulo.png}
    \end{figure}
\end{multicols}
\end{enumerate}

\section*{\underline{Espejos y Lentes}}
\begin{enumerate}

\item Una persona de altura $h$ desde sus pies hasta sus ojos se para frente a un espejo plano, el cual parte a la altura de los ojos de la persona. ¿Cuál es el largo del espejo para que la persona sea capaz de mirar, a más poder, sus zapatos?

\begin{figure}[H]
    \centering
    \includegraphics[width=0.2\linewidth]{Imágenes/clases/persona.png}
\end{figure}

\item Para un espejo cóncavo de longitud focal $f$, ¿cuál debe ser la distancia $d_0$ del objeto para que la distancia de la imagen sea igual a esta? Dibuje los rayos principales, ¿es una imagen real o virtual?¿derecha verticalmente o invertida?

\item Cuando un objeto está a cierta distancia de un espejo cóncavo, el aumento de la imagen es $M_1$. Si el objeto se mueve una distancia $\ell$ de su ubicación original, el aumento de la imagen pasa a ser $M_2$. ¿Cuál es la distancia focal de este espejo?

\item Considere un sistema óptico compuesto por una lente convexa con longitud focal $f_L = \SI{0.02}{\m}$ y un espejo convexo $M$ de radio $R_M = \SI{0.12}{\m}$. Si la distancia entre el lente y el espejo es $d=\SI{0.04}{\m}$, determine la ubicación de la imagen de un objeto ubicado a $\SI{0.03}{m}$ de la lente

\begin{figure}[H]
    \centering
    \includegraphics[width=0.3\linewidth]{Imágenes/clases/lens-mirror.png}
\end{figure}

\item Como se muestra en la figura, tres lentes biconvexos idénticos de distancia focal $f$ son alineados y separados una distancia $f$ entre ellos. Usando rayos principales, encuentre la posición y la magnificación de la imagen resultante si se ubica un objeto a una distancia $f/2$ del lente ubicado más a la izquierda

\begin{figure}[H]
    \centering
    \includegraphics[width=0.35\linewidth]{Imágenes/clases/three-lenses.png}
\end{figure}

\item Considere el siguiente sistema óptico compuesto por una lente convergente y una divergente. La distancia focal de los lentes son, respectivamente, $f_1=\SI{50}{\mm}$ y $f_2 = \SI{-100}{\mm}$. \textbf{(a)} Determine la distancia $d$ entre el lente divergente y el sensor si el sistema enfoca un objeto que se encuentra a una distancia $\SI{0.5}{\m}$ del lente $L_1$. \textbf{(b)} ¿Cuál es la magnificación de la imagen formada por este sistema?

\begin{figure}[H]
    \centering
    \includegraphics[width=0.35\linewidth]{Imágenes/clases/sensor.png}
\end{figure}

\end{enumerate}
\end{document}