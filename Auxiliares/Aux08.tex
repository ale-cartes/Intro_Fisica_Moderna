\documentclass[letterpaper,11pt]{article}
\oddsidemargin -1.0cm \textwidth 17.5cm

\usepackage[utf8]{inputenc}
\usepackage[activeacute,spanish]{babel}
\usepackage{amsfonts,setspace}
\usepackage{amsmath}
\usepackage{amssymb, amsmath, amsthm}
\usepackage{comment}
\usepackage{float}
\usepackage{amssymb}
\usepackage{dsfont}
\usepackage{anysize}
\usepackage{multicol}
\usepackage{enumerate}
\usepackage{graphicx}
\usepackage[left=1.5cm,top=1.5cm,right=1.5cm, bottom=1.7cm]{geometry}
\setlength\headheight{1.5em} 
\usepackage{fancyhdr}
\usepackage{multicol}
\usepackage{hyperref}
\usepackage{wrapfig}
\usepackage{subcaption}
\pagestyle{fancy}
\fancyhf{}
\renewcommand{\labelenumi}{\normalsize\bfseries P\arabic{enumi}.}
\renewcommand{\labelenumii}{\normalsize\bfseries (\alph{enumii})}
\renewcommand{\labelenumiii}{\normalsize\bfseries \roman{enumiii})}

\begin{document}

\fancyhead[L]{\itshape{Facultad de Ciencias F\'isicas y Matem\'aticas}}
\fancyhead[R]{\itshape{Universidad de Chile}}

\begin{minipage}{11.5cm}
    \begin{flushleft}
        \hspace*{-0.6cm}\textbf{FI1100-6 Introducción a la Física Moderna}\\
        \hspace*{-0.6cm}\textbf{Profesor:} Diego Mardones\\
        \hspace*{-0.6cm}\textbf{Auxiliares:} Gabriel O\textsc{\char13}Ryan, Camila Sepúlveda, Alejandro Silva\\
        \hspace*{-0.6cm}\textbf{Ayudante:} Sebastián Vargas
    \end{flushleft}
\end{minipage}

\begin{picture}(2,3)
    \put(366, 10){\includegraphics[scale=0.9]{Imágenes/logo/dfi-fcfm.pdf}}
\end{picture}

\begin{center}
	\LARGE\textbf{Auxiliar \# 8 :}\\
	\Large{Interferencia y difracción}
\end{center}

\vspace{-1cm}
\begin{enumerate}\setlength{\itemsep}{0.4cm}

\rfoot[]{pág. \thepage}

\item[]

\item \textbf{Interferencia entre fuentes de sonido}\\
Dos altavoces emiten con la misma frecuencia  y longitud de onda $\lambda=2,0$m. Estos se encuentran separados por una distancia d = 6.0m.
Una persona se encuentra a 8.0 metros frente al altavoz derecho (Considere a la persona como un punto).

a) Calcule la distancia que debe recorrer la onda generada por cada altavoz hasta la persona.

b) ¿Que tipo de interferencia sufrirán las ondas en la posición de la persona?.

c) Siguiendo el problema anterior, luego de un tiempo la persona ya se encuentra agotada del ruido (o silencio) y solo quiere estar en un lugar silencioso (o ruidoso).
Considerando que la persona se quiere mover lo menos posible y solo alejándose del altavoz derecho. ¿Cuanto debe moverse para cambiar la situación anterior?.


\item \textbf{Experimento de la rendija doble}\\
A través de dos ranuras muy angostas separadas por una distancia de 0.200 mm se hace pasar luz coherente con longitud de onda de 400 nm, y el patrón de interferencia se observa en una pantalla ubicada a 4.00 m de las
ranuras. (Asuma el mismo ancho entre franja de interferencia).

a) (Resumen del calculo para la rendija) ¿Para los valores propuestos, que tan validos son para cumplir las aproximaciones realizadas en el experimento?

b) ¿Cuál es el ancho (en mm) del máximo central de interferencia?


\item \textbf{Experimento con una rendija}\\
Se hace pasar un láser con longitud de onda 630 nm a través de una ranura angosta y se observa un patrón de difracción en una pantalla a 8 m de distancia. Se encuentra que, en la pantalla, la distancia entre los centros de los primeros mínimos fuera de la franja brillante central es de 32 mm. 

a) ¿Cuál es el ancho de la ranura?


\end{enumerate}
\end{document}